\section{企業型年金の開始}


%1節 目的
\subsubsection{DC法の目的}
\barquo{
  以下は、確定拠出年金法第1条(目的)を引用したものであるが、
  いくつかの箇所に誤りが含まれている。
  誤りの含まれている箇所を列挙し、正しい内容に修正しなさい。

  \begin{quote}
    \textbf{確定拠出年金法}

    \;\;\;\; \textbf{(目的)}

    \textbf{第一条} \;\;
    この法律は、少子高齢化の進展、産業構造の変化等の社会経済情勢の変化にかんがみ、
    事業主が従業員と給付の内容を約し、
    高齢期において従業員がその内容に基づいた給付を受けることができるようにするため、
    確定拠出年金について必要な事項を定め、国民の高齢期における所得の確保に係る自主的な努力を支援し、
    もって公的年金の給付と相まって国民の生活の安定と福祉の向上に寄与することを目的とする。
  \end{quote}

  \rightline{(自作)}
}

\begin{itembox}[l]{\textgt{ポイント}}
  DB法とDC法の目的は非常によく似ているのでセットで覚えるとよい。

  \begin{quote}
    \textbf{確定給付企業年金法}

    \textbf{第一条} \;\;
    この法律は、少子高齢化の進展、\textcolor{red}{産業構造の変化}等の社会経済情勢の変化にかんがみ、
    \textcolor{red}{事業主が従業員と給付の内容を約し、高齢期において従業員がその内容に基づいた給付}を受けることが
    できるようにするため、\textcolor{red}{確定給付企業年金}について必要な事項を定め、
    国民の高齢期における所得の確保に係る自主的な努力を支援し、
    もって公的年金の給付と相まって国民の生活の安定と福祉の向上に寄与することを目的とする。
  \end{quote}

  \begin{quote}
    \textbf{確定拠出年金法}

    \textbf{第一条} \;\;
    この法律は、少子高齢化の進展、\textcolor{red}{高齢期の生活の多様化}等の社会経済情勢の変化にかんがみ、
    \textcolor{red}{個人又は事業主が拠出した資金を個人が自己の責任において運用の指図を行い、
    高齢期においてその結果に基づいた給付}を受けることができるようにするため、
    \textcolor{red}{確定拠出年金}について必要な事項を定め、国民の高齢期における所得の確保に係る自主的な努力を支援し、
    もって公的年金の給付と相まって国民の生活の安定と福祉の向上に寄与することを目的とする。
  \end{quote}
\end{itembox}

\begin{sol}
  \;

  \begin{itemize}
    \item (誤)産業構造の変化 $\Longrightarrow$ (正)高齢期の生活の多様化
    \item (誤)事業主が従業員と給付の内容を約し、
      高齢期において従業員がその内容に基づいた給付を受けることができるようにするため 
      
      $\Longrightarrow$ 
      (正)個人又は事業主が拠出した資金を個人が自己の責任において運用の指図を行い、
      高齢期においてその結果に基づいた給付を受けることができるようにするため
  \end{itemize}
\end{sol}

\newpage

%5節 簡易企業型年金
\subsubsection{簡易企業型年金}
\barquo{
  確定拠出年金法第3条第5項に規定されている簡易企業型年金と
  それ以外の通常の企業型年金の違いについて、
  加入者の範囲、掛金の算定方法、加入者掛金の額の選択肢、
  対象運用方法の数、対象の企業規模の観点から記載しなさい。

  \rightline{(自作)}
}

\begin{itembox}[l]{\textgt{ポイント}}
  それぞれ以下の法令に記載がある。
  \begin{itemize}
    \item 加入者の範囲:法第3条第5項第1号
    \item 掛金の算定方法:令第10条の3
    \item 加入者掛金の額の選択肢:解釈第1-3(3)
    \item 対象運用方法の数:令第16条第2項
    \item 対象の企業規模:法第3条第5項第2号
  \end{itemize}
\end{itembox}

\begin{sol}
  \;

  \begin{center}
    \begin{tabular}{l|c|c} 
      \hline
      & 簡易DC & 通常のDC  \\
      \hline \hline
      加入者の範囲 & \begin{tabular}{c}
        第1号等厚生年金被保険者全員 \\
        (一定の資格を定めることは\textcolor{red}{できない})
        \end{tabular} 
       & \begin{tabular}{c}
        第1号等厚生年金被保険者 \\
        (一定の資格を定めることは\textcolor{red}{できる})
        \end{tabular} \\ \hline
      掛金の算定方法 & \textcolor{red}{定額のみ} & 定額、定率、定額$+$定率 \\ \hline
      加入者掛金の額の選択肢 & \textcolor{red}{1つでも可} & 必ず\textcolor{red}{2つ以上} \\ \hline
      対象運用方法の数 & \textcolor{red}{2本以上}35本以下 & \textcolor{red}{3本以上}35本以下 \\ \hline
      対象の企業規模 & 企業型年金加入者資格を有する者が\textcolor{red}{300人以下} & - \\ \hline
    \end{tabular}
  \end{center}
\end{sol}

\newpage

