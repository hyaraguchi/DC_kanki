\section{制度移換・終了}

%1節 他制度からの資産移換
\subsubsection{退職手当制度からの移換}
\barquo{
  以下の記述の正誤を判定し、誤っている場合はその理由を簡記しなさい。

  \;

  平成31 年 1 月に退職給与規程を改正することにより、事業主が企業型年金の資産管理機関
  へ資産を移換する場合、移行年度から、移行年度の翌年度から起算して三年度以上七年度以
  内の企業型年金規約で定める年度までの各年度に均等に分割して移換する必要があり、移行
  年度は退職給与規程の改正が行われた日の属する年度とする必要がある。

  \rightline{引用元:年金1 2018 問2(2)\textcircled{1}}
}

\begin{itembox}[l]{\textgt{ポイント}}
  令第22条第1項第5号。

  退職手当制度はDBなどのように、どこかに明示的に資産が積み立てられているわけではないので、
  移換すべき資産の額の確定に時間を要する可能性がある。
\end{itembox}

\begin{sol}
  \;

  誤。

  年度末(3月31日)から3か月以内に移管資産の額を確定することが困難であると認められる場合は、
  当該年度の翌年度、つまり平成31年4月からを移行年度とすることができる。

\end{sol}

\newpage