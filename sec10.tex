\section{ポータビリティ}


%8節 個人別管理資産の移換時の通知と説明義務
\subsubsection{個人別管理資産の移換時の説明義務\textcircled{1}}
\barquo{
  個人別管理資産に企業型年金加入者掛金(以下「本人拠出相当額」 )が含まれる企業型年金
  加入者が企業型年金加入者の資格を喪失するとともに確定給付企業年金の加入者の資格を取得
  し、当該確定給付企業年金へ個人別管理資産の移換を行うときに、 『確定拠出年金法並びにこ
  れに基づく政令及び省令について(法令解釈) 』に定める、本人拠出相当額の課税に関して事
  業主が当該資格喪失者に対して十分説明することとされている事項について簡記しなさい。
  \rightline{引用元:年金1 2020 問2(2)\textcircled{1}}
}

\begin{itembox}[l]{\textgt{ポイント}}
  企業型DC加入者が資格喪失したときに事業主が説明すべき事項は、解釈第11-1.に以下の4つが挙げられている。

  \begin{enumerate}
    \item 個人別管理資産の移換の申出は、資格喪失した日の属する月の翌月から起算して\textcolor{red}{6ヶ月以内}に行うこと。
    \item 上記の申出を行わない場合は、以下のいずれかの取り扱いがされること。
    \begin{enumerate}
      \item 他の企業型DCの加入者となる場合は、その企業型DCに個人別管理資産が移換される。
      \item 個人型DCの加入者となる場合は、その個人型DCに個人別管理資産が移換される。
      \item 上記のいずれにも該当しない場合は、個人別管理資産は連合会に自動移換される。
      なお、連合会移換者である間は運用されず、管理手数料が引き落とされる。
      その際、当該期間は通算加入者期間に算入されないため、老齢給付金の支払い開始時期が遅くなる可能性がある。
    \end{enumerate}
    \item DBの加入者になる場合は、資格喪失した日の属する月の翌月から起算して\textcolor{red}{6ヶ月以内}であれば、
      DBへ個人別管理資産を移換できること。
      また、個人別管理資産が連合会に自動移換されている者(2.(c)の者)が、DBの加入者となった場合、DBへ資産移換できること。

      なお、DBの本人拠出相当額は\textcolor{red}{拠出時に課税}、\textcolor{red}{給付時に非課税}の取り扱いであるが、
      移換する個人別管理資産に企業型DCの本人拠出相当額が含まれていても、DBの本人拠出相当額としての取り扱いではなく、
      \textcolor{red}{給付時に課税}されることも説明しなければならない。
    
    \item 企業型DCからDBや中退共に移換する場合、DCに加入していた期間が移換先の制度設計に合わせた期間に調整される可能性があること。
  \end{enumerate}

  この問題は「本人拠出額の課税」に関してのみ問うているため、3.のなお書以降をまとめればよい。

\end{itembox}

\begin{sol}
  \;
  
  企業型DCの本人拠出額は拠出時に課税されないため、確定給付企業年金に移換する個人別管理資産の中に
  本人拠出額が含まれていても、給付時に課税される。
  確定給付企業年金の本人拠出額の取り扱いが、拠出時に課税、給付時に非課税であるため、
  両者を混同しないように資格喪失者に対して十分説明する必要がある。
\end{sol}

\newpage

\subsubsection{個人別管理資産の移換時の説明義務\textcircled{2}}
\barquo{
  企業型年金の加入者が資格を喪失した場合に事業主が当該資格喪失者に対して十分説明する
  こととされている事項のうち、「個人別管理資産がある企業型年金加入者が資格を喪失し、資
  格喪失日の属する月の翌月から起算して6月以内に他の企業型年金等へ個人別管理資産の移換
  を行う旨の申出を行わない場合の取扱い」について、簡記しなさい。

  \rightline{引用元:年金1 2019 問2(4)\textcircled{2}}
}

\begin{itembox}[l]{\textgt{ポイント}}
  企業型DC加入者が資格喪失したときに事業主が説明すべき事項は、解釈第11-1.に以下の4つが挙げられている。
  (再掲)

  \begin{enumerate}
    \item 個人別管理資産の移換の申出は、資格喪失した日の属する月の翌月から起算して\textcolor{red}{6ヶ月以内}に行うこと。
    \item 上記の申出を行わない場合は、以下のいずれかの取り扱いがされること。
    \begin{enumerate}
      \item 他の企業型DCの加入者となる場合は、\textcolor{red}{その企業型DCに個人別管理資産が移換される。}
      \item 個人型DCの加入者となる場合は、\textcolor{red}{その個人型DCに個人別管理資産が移換される。}
      \item 上記のいずれにも該当しない場合は、個人別管理資産は\textcolor{red}{連合会に自動移換される。}
      なお、連合会移換者である間は\textcolor{red}{運用されず、管理手数料が引き落とされる。}
      その際、\textcolor{red}{当該期間は通算加入者期間に算入されないため、老齢給付金の支払い開始時期が遅くなる可能性がある。}
    \end{enumerate}
    \item DBの加入者になる場合は、資格喪失した日の属する月の翌月から起算して6ヶ月以内であれば、
      DBへ個人別管理資産を移換できること。
      また、個人別管理資産が連合会に自動移換されている者(2.(c)の者)が、DBの加入者となった場合、DBへ資産移換できること。

      なお、DBの本人拠出相当額は拠出時に課税、給付時に非課税の取り扱いであるが、
      移換する個人別管理資産に企業型DCの本人拠出相当額が含まれていても、DBの本人拠出相当額としての取り扱いではなく、
      給付時に課税されることも説明しなければならない。
    
    \item 企業型DCからDBや中退共に移換する場合、DCに加入していた期間が移換先の制度設計に合わせた期間に調整される可能性があること。
  \end{enumerate}

  この問題は移換の申出をしなかったケースを問うているため、2.の内容を答えればよい。

\end{itembox}

\begin{sol}
  \;

  説明すべき事項を以下に列挙する。

  \begin{itemize}
    \item 他の企業型年金の加入者となる場合は、その企業型年金に個人別管理資産が移換されること。
    \item 個人型年金の加入者となる場合は、その個人型年金に個人別管理資産が移換されること。
    \item 上記のいずれにも該当しない場合は、個人別管理資産は国民年金連合会に移換されること。
    なお、連合会移換者である間は運用されず、管理手数料が引き落とされること。
    また、その期間は通算加入者期間に算入されないため、老齢給付金の支給開始時期が遅くなる可能性があること。
  \end{itemize}
\end{sol}

\newpage

