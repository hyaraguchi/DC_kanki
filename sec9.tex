\section{個人型掛金}

%2節 中小事業主掛金
\subsubsection{中小事業主掛金}
\barquo{
  確定拠出年金法施行令第29 条第4 号に定める、中小事業主が個人型年金加入者の掛金に上乗
  せして中小事業主掛金を拠出することを定める場合に満たすべき要件を簡記しなさい。

  \rightline{引用元:年金1 2021 問2(2)\textcircled{2}}
}

\begin{itembox}[l]{\textgt{ポイント}}
  令第29条第1項第4号。いわゆる\textcolor{red}{iDeCo+}の掛金の要件を列挙すればよい。
\end{itembox}

\begin{sol}
  \;

  中小事業主掛金を拠出する場合は次に満たす要件を満たす必要がある。

  \begin{itemize}
    \item 中小事業主掛金の額の決定または変更の方法は、特定の者に不当に差別的なものでないこと。
    \item 中小事業主掛金について、前納および追納することができないものであること。
    \item 中小事業主掛金の額は、中小事業主掛金を拠出することが困難である場合を除き、
      個人型掛金拠出単位期間につき1回に限り変更することができるものであること。
  \end{itemize}

\end{sol}

\begin{shadebox}
  iDeCo+の案内サイト

  \url{https://www.mhlw.go.jp/stf/seisakunitsuite/bunya/0000194195.html}
\end{shadebox}

\newpage

\subsubsection{中小事業主掛金:一定の資格}
\barquo{
  中小事業主が個人型年金加入者の掛金に上乗せして拠出する中小事業主掛金の拠出対象とな
  る者については「一定の資格」を定めることができるが、『確定拠出年金法並びにこれに基づく
  政令及び省令について(法令解釈)』に記載されている「一定の資格」として定めることのでき
  るものをすべて簡記しなさい。

  \rightline{引用元:年金1 2021 問2(2)\textcircled{3}}
}

\begin{itembox}[l]{\textgt{ポイント}}
  解釈第2-2.(1)。

  「一定の資格」として定めることができる要件をまとめた表は以下の通り。

  \;

  \begin{center}
    \begin{tabular}{l|ccc} 
      \hline
      & DBの加入者 & 企業型DCの加入者 & 中小事業主掛金の拠出の対象 \\
      \hline \hline
      一定の職種 & ◯ & ◯ & ◯ \\
      一定の勤続期間 & ◯ & ◯ & ×\\
      一定の年齢 & ◯ & ◯ & ◯ \\
      希望する者 & ◯ & ◯ & ×\\
      休職等期間中でない者 & ◯ & × & ×\\
    \end{tabular}
  \end{center}

  $\rightarrow$ \textcolor{red}{「一定の職種」と「一定の年齢」}についてだけ書けばよい。
\end{itembox}

\begin{sol}
  \;

  \begin{itemize}
    \item 「一定の職種」に属する加入者のみを中小事業主掛金の拠出対象者とすること。
    \item 当該厚生年金適用事業所に使用される期間のうち、「一定の勤続期間以上(または未満)」の
    加入者のみを中小事業主掛金の拠出対象者とすること。
  \end{itemize}

\end{sol}


\newpage