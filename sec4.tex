\section{他制度掛金相当額}

%1節 他制度掛金相当額の見直し
\subsubsection{他制度掛金相当額:加入者掛金の取り扱い}
\barquo{
  通知「確定拠出年金における他制度掛金相当額及び共済掛金相当額の算定方法について(令
  和3 年9 月1 日年企発第0901 第2 号)」に定める、加入者が掛金の一部を負担している場合
  の他制度掛金相当額及び共済掛金相当額にかかる取扱いについて簡潔に入力しなさい。な
  お、 「確定給付企業年金制度の場合」と「確定給付企業年金制度以外の他制度の場合」のそれ
  ぞれについて入力すること。また、制度内容に応じた具体的な算定方法には触れなくてよ
  い。 (250 字以内)

  \rightline{引用元:年金1 2023 問2(2)(イ) }
}

\begin{itembox}[l]{\textgt{ポイント}}
  算定省令第5条の理解を問う問題。

  DBでは、他制度掛金相当を算定する際、加入者負担掛金を\textcolor{red}{零}とみなして算定するが、
  DB以外の他制度の場合は、\textcolor{red}{加入者負担掛金も含めて算定する}必要がある。

  実務的には、加入者が掛金を負担している給付区分にかかる他制度掛金相当額を0円にするなど、
  合理的な方法で算定するようにしなければならない。
\end{itembox}

\begin{sol}
  \;

  「確定給付企業年金制度の場合」

   加入者が負担する掛金は零であるものとして算定する。

  「確定給付企業年金制度以外の他制度の場合」

  加入者が負担する掛金も含めて算定する。
\end{sol}

\newpage

%2節 他制度掛金相当額の算定
\subsubsection{他制度掛金相当額の算定にかかる経過措置}
\barquo{
  「確定拠出年金における他制度掛金相当額及び共済掛金相当額の算定に関する省令(令和3 年厚
  生労働省令第150 号)」において定められている、2024 年(令和6 年)12 月1 日前を計算基
  準日とする財政計算の結果に基づいて掛金の額を算定する事業主等の確定給付企業年金の加
  入者に係る他制度掛金相当額に関する経過措置について簡潔に入力しなさい。なお、リスク
  分担型企業年金以外である場合についてのみ入力すること。

  \rightline{引用元:年金1 2022 問2(2)(イ) }
}

\begin{itembox}[l]{\textgt{ポイント}}
  算定省令附則第2条。2024年12月1日以前を計算基準日とする財政計算で他制度掛金を算定する場合、
  経過措置として、簡易型DBと同じ方法で他制度掛金を算定できる。
  解答には、簡易型DBの他制度掛金算定方法を入力すればよい。
\end{itembox}

\begin{sol}
  \;

  直近の財政計算の基準日における当該財政計算の結果に基づく標準掛金額を当該財政計算の
  計算基準日における加入者の数で除した額を1月あたりの額に換算した額とすることができる。
\end{sol}

\newpage

\subsubsection{休職等の取り扱い}
\barquo{
  通知「確定拠出年金における他制度掛金相当額及び共済掛金相当額の算定方法について(令
  和3 年9 月1 日年企発第0901 第2 号)」において定められている、一部の加入者の確定給付
  企業年金の掛金拠出がない場合(※)における、当該加入者に係る他制度掛金相当額の取り
  扱いについて簡潔に入力しなさい。

  (※)確定給付企業年金に加入している休職者であって、掛金の拠出を中断する取扱いや一
  定の年齢以降を給付の額の算定の基礎としていない等によるもの。

  \rightline{引用元:年金1 2022 問2(2)(ウ) }
}

\begin{itembox}[l]{\textgt{ポイント}}
  休職等の取り扱いについては、Q\&A15,16に記載がある。

  一部の加入者について掛金の拠出がない場合も、他制度掛金相当額は零円ではなく、
  \textcolor{red}{他の加入者と同額}を設定しなければならない。
\end{itembox}

\begin{sol}
  \;

  掛金拠出や給付の額の算定の基礎の取り扱いにかかわらず、確定給付企業年金の加入者である以上、
  当該確定給付企業年金の他の加入者と同じ金額を設定する必要がある。
\end{sol}

\newpage