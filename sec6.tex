\section{給付}


%2節 老齢給付金
\subsubsection{老齢給付金の支給要件:通算加入者期間}
\barquo{
  確定拠出年金法第33条第1項に定める、老齢給付金の支給の請求に関する通算加入者等期
  間の要件を簡潔に入力しなさい。ただし、以下の取扱いについては触れなくてよい。 (250 字
  以内)

  \;

  <触れなくてよい取扱い>
  
  企業型年金加入者であった者のうち一定の年齢要件を満たす者について、通算加入者等期
  間を有しない場合であっても企業型年金加入者となった日などから起算して5年を経過し
  た日から請求できる。

  \rightline{引用元:年金1 2023 問2(2)(ア) }
}

\begin{itembox}[l]{\textgt{ポイント}}
  老齢給付金の支給を請求できるようになる年齢は、\textcolor{red}{通算加入者期間の年数(あるいは月数)}によって異なる。
\end{itembox}

\begin{sol}
  \;

  通算加入者期間の要件は、年齢に応じて以下の通り。

  \begin{table}[h]
    \label{table:通算加入者期間の要件}
    \centering
    \begin{tabular}{cc} 
      \hline
      請求可能年齢 & 通算加入者期間   \\
      \hline \hline
      60歳以上61歳未満 & 10年  \\
      61歳以上62歳未満 & 8年   \\
      62歳以上63歳未満 & 6年\\
      63歳以上64歳未満 & 4年 \\
      64歳以上65歳未満 & 2年  \\
      65歳以上 & 1月 \\
      \hline
    \end{tabular}
  \end{table}
\end{sol}

\begin{shadebox}
  <触れなくてよい取扱い>は、令和4年5月1日施行で加入可能年齢が引き上げになったことを受けて、
  60歳以上に初めてDCの加入者になった場合など、60歳以上75歳未満の方で通算加入者期間がない人に対する措置。
\end{shadebox}

\newpage