\section{企業型年金加入者等}

%1節 企業型年金加入者
\subsubsection{加入可能年齢の変更}
\barquo{
  「年金制度の機能強化のための国民年金法等の一部を改正する法律」(令和 2 年6 月5 日公
  布)により確定拠出年金の加入可能要件の見直し等が実施されることに関し、次の(a)およ
  び(b)の内容を簡記しなさい。

  \begin{enumerate}
    \renewcommand{\labelenumi}{(\alph{enumi})}
    \item 令和 4 年 5 月1 日施行の企業型年金における加入可能年齢の変更について
    \item 令和 4 年 5 月1 日施行の個人型年金における加入可能年齢の変更について
  \end{enumerate}

  \rightline{引用元:年金1 2020 問2(2)\textcircled{3}}
}

\begin{itembox}[l]{\textgt{ポイント}}
  令和2年6月の法改正による加入要件緩和に関する問題。

  \begin{center}
    \begin{tabular}{l|cc} 
      \hline
      & 法改正前 & 法改正後  \\
      \hline \hline
      企業型年金(法第9条) & 
        \begin{tabular}{c}
        厚生年金保険の被保険者 \\
        のうち65歳未満の者
        \end{tabular} 
      & \textcolor{red}{厚生年金保険の被保険者} \\ \hline
      個人型年金(法第62条) & 
        \begin{tabular}{c}
        国民年金の被保険者 \\
        のうち60歳未満の者
        \end{tabular}
      & \textcolor{red}{国民年金の被保険者}\\
    \end{tabular}
  \end{center}
\end{itembox}

\begin{sol}
  \;

  \begin{enumerate}
    \renewcommand{\labelenumi}{(\alph{enumi})}
    \item 令和4年5月1日施行以前は、厚生年金保険の被保険者のうち65歳未満の者しか企業型年金に
      加入できなかったが、施行後は厚生年金保険の被保険者であれば加入者となることができるようになった。
    \item 令和4年5月1日施行以前は、国民年金の被保険者のうち60歳未満の者しか個人型年金に
    加入できなかったが、施行後は国民ねんきんの被保険者であれば個人型年金に加入可能になった。
  \end{enumerate}
\end{sol}

\newpage

\subsubsection{一定の資格}
\barquo{
  企業型年金加入者となることについて「一定の資格」として定めることができる4 つの資格
  および、当該「一定の資格」を定める場合に、企業型年金加入者とならない従業員に対して基
  本的には適用されていることが必要とされる措置(代替措置)について簡記しなさい。

  \rightline{引用元:年金1 2019 問2(4)\textcircled{1}}
}

\begin{itembox}[l]{\textgt{ポイント}}
  解釈第1-1。
  一定の資格として定めることができる要件は以下の通り。
  \begin{enumerate}
    \item 一定の職種
    \item 一定の勤続期間
    \item 一定の年齢
    \item 希望するもの
  \end{enumerate}
  加入除外された者へは以下の代替措置を設ける必要がある。
  \begin{itemize}
    \item 「一定の職種」「一定の勤続期間」
    \begin{itemize}
      \item イメージはすでに他制度がある場合の代替措置。
      \item 厚生年金基金、DB、退職一時金制度(前払い退職金制度も含む)に加入し続けることが代替措置になる。
    \end{itemize}
    \item 「一定の年齢」「希望する者」
    \begin{itemize}
      \item どちらかといえば、新たにDCを実施する場合の代替措置。
      \item 退職一時金制度(前払い一時金制度も含む)に加入し続けることが代替措置になる。
      \item 「希望する者」の場合は、DBに加入することでもよい。
    \end{itemize}
  \end{itemize}
\end{itembox}

\begin{sol}
  \;

  企業型年金加入者となることについて定めることができる資格とは、
  \begin{enumerate}
    \item 一定の職種
    \item 一定の勤続年数
    \item 一定の年齢
    \item 希望する者
  \end{enumerate}
  の四つである。

  このうち1、2については、厚生年金基金(加算部分)、確定給付企業年金または退職一時金制度
  (前払い一時金制度も含む)が適用されていることが代替措置となる。
  また、3、4については、確定給付企業年金(4の場合のみ)または退職一時金制度
  (前払い一時金制度も含む)が適用されていることが代替措置となる。


\end{sol}

\newpage


%2節 資格取得の時期
\subsubsection{資格取得の時期}
\barquo{
  確定拠出年金法第10条に記載されている企業型年金加入者の資格取得時期をすべて列挙しなさい。

  \rightline{(自作)}
}

\begin{itembox}[l]{\textgt{ポイント}}
  DBの資格取得時期とまったく同じ。
\end{itembox}

\begin{sol}
  \;

  以下の4点。
  \begin{itemize}
    \item 実施事業所に使用されるに至ったとき。
    \item その使用されている事業所もしくは事務所または船舶が、実施事業所になったとき。
    \item 実施事業所に使用されている者が、厚生年金保険の被保険者となったとき。
    \item 実施事業所に使用されている者が、企業型年金規約に定める資格を取得したとき。
  \end{itemize}
\end{sol}

\newpage
